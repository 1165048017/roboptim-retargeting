\documentclass{article}

\usepackage{manfnt}

\makeatletter
\def\hang{\hangindent\parindent}
\def\d@nger{\medbreak\begingroup\clubpenalty=10000
  \def\par{\endgraf\endgroup\medbreak} \noindent\hang\hangafter=-2
  \hbox to0pt{\hskip-\hangindent\dbend\hfill}}
\outer\def\danger{\d@nger}
\makeatother

\title{Motion Retargeting}
\author{}
\date{}

\begin{document}
\maketitle

The goal of this problem is to adapt a trajectory captured using a
motion capture system to a robot.

\section{Notations}

A trajectory is composed of $M$ frames. The robot is composed of $N$
joints. $q_i^j$ denotes the position of the $i$-th joint at the $j$-th
frame.


\section{Joint Optimization}

The vector of optimization contains the joint value for each frame of
the trajectory. It is noted $\mathbf{x}$ and can be illustrated as
follow:


\begin{center}
  \begin{tabular}{|c c c | c | c c c|}
    \hline
    $q_0^0$ & \ldots & $q_N^0$ & \ldots & $q_0^M$ & \ldots & $q_N^M$\\
    \hline
  \end{tabular}
\end{center}

$x^j$ is the configuration vector for the $j$-th frame.


\subsection{Problem statement}

\subsection{Cost function}

Let $F$ be the problem cost function.

\begin{equation}
  F(\mathbf{x}) = {LDE}(\mathbf{x}) + \alpha {Acc}(\mathbf{x})
\end{equation}


$\alpha$ is a weight on the acceleration term. Its default value is
$10^{-6}$.

The $LDE$ function denotes the Laplacian Deformation Energy while the
Acc term denotes the acceleration energy.

\paragraph{Laplacian Deformation Energy}

The Laplacian Deformation Energy associated with a marker is:

\begin{equation}
  LDE(v_i^j) = \frac{1}{2} \| \delta_i^j - L(v_i^j) \|^2
\end{equation}


$L(v_i^j)$ is the Laplacian Coordinate of the marker $i$ at frame $j$.


\begin{equation}
  L(v_i^j) = v_i^j - \sum_{l\in N(v_i^j)} w_{(i,l)}^j v_l^j
\end{equation}


$N(v_i^j)$ are the neighbors of the marker $v_i^j$
(one-ring). $w_{(i,l)}^j$ is the weight of marker $i$ in frame $j$
which is inversely proportional to the distance of $v_i^j$ and $v_l^j$
in the original motion. These weights are precomputed at the beginning
of the motion.


Each marker position can be computed from its attached joint position
using the JointToMarker function:

\begin{equation}
  JointToMarker(q_i^j) = offset + DirectGeometry(\mathbf{x}^j, i)
\end{equation}

This function relies on the $DirectGeometry$ function which computes
the position of each body attached to each joint for a particular
frame $j$ and a particular
configuration. $DirectGeometry(\mathbf{x}^j, i)$ is the position of
the body $i$ for the configuration $\mathbf{x}^j$ (i.e. the
configuration at the $j$-th frame).


Therefore, the previous equation can be rewritten as:

\begin{equation}
  LDE(q_i^j) = \frac{1}{2} \| \delta_i^j - L(JointToMarker(q_i^j)) \|^2
\end{equation}


The previous equations are considering the Laplacian Deformation
Energy for one frame. The final term of the cost function is computed
by summing each frame Laplacian Deformation Energy associated with
each joint:


\begin{equation}
  LDE(\mathbf{x}) = \sum_{j \in M} \sum_{i \in N}  LDE(x_i^j)
\end{equation}


\danger There may be zero or several markers associated to some
joints. Some are also of little interest and ignored to preserve
computational power (for instance the face joints).


By expanding the previous equation and replacing the symbols, one can
obtain:


\begin{equation}
  LDE(\mathbf{x}) = \frac{1}{2}  \sum_{j \in M} \sum_{i \in N} \big\| \delta_i^j - \big( JointToMarker(\mathbf{x}_i^j) - \sum_{l\in N(v_i^j)} w_{(i,l)}^j JointToMarker(\mathbf{x}_l^j) \big) \big\|^2
\end{equation}


\paragraph{Acceleration Energy}


The joint acceleration is considered as follow:


\begin{equation}
  {Acc}(\mathbf{x}) = \ddot{x}
\end{equation}


\danger In the original paper and Choreonoid implementation, the
marker acceleration is considered and not the joint one.


\subsection{Constraints}

\paragraph{Bone Length}

The bone length constraint is expressed as:

\begin{equation}
  BoneLength(p_1, p_2) = (\| p_1 - p_2 \| - l_e)^2
\end{equation}

\paragraph{Position}

\begin{equation}
  Position(m_i^j) = T
\end{equation}

Where $T = (x, y, z)$ is a particular reference point.


The constraint can be evaluated through the use of $JointToMarker$:

\begin{equation}
  Position_{(i,j)}(\mathbf{x}) = Position(JointToMarker(\mathbf{x}^j, i))
\end{equation}

\paragraph{ZMP}

The ZMP can be computed for a particular value of $\mathbf{q}, \mathbf{\dot{q}}, \mathbf{\ddot{q}}$.

\begin{equation}
  ZMP(\mathbf{q}, \mathbf{\dot{q}}, \mathbf{\ddot{q}})
\end{equation}

An estimation of $\mathbf{\dot{q}}$ and $\mathbf{\ddot{q}}$ can be
computed from $\mathbf{x}$ using numerical differentiation:

\begin{equation}
  ZMP_j(\mathbf{x}^j, \mathbf{\dot{x}}^j, \mathbf{\ddot{x}}^j)
\end{equation}


\paragraph{Torque}

The torque values can be computed for a particular value of
$\mathbf{q}, \mathbf{\dot{q}}, \mathbf{\ddot{q}}$.

\begin{equation}
  Torque(\mathbf{q}, \mathbf{\dot{q}}, \mathbf{\ddot{q}})
\end{equation}

An estimation of $\mathbf{\dot{q}}$ and $\mathbf{\ddot{q}}$ can be
computed from $\mathbf{x}$.


\begin{equation}
  Torque_j(\mathbf{x}^j, \mathbf{\dot{x}}^j, \mathbf{\ddot{x}}^j)
\end{equation}


\section{Spline Joint Optimization}

TO BE DONE


\end{document}
